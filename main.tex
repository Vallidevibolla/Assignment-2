\documentclass[journal,12pt,twocolumn]{IEEEtran}

\usepackage{setspace}
\usepackage{gensymb}

\singlespacing


\usepackage[cmex10]{amsmath}

\usepackage{amsthm}

\usepackage{mathrsfs}
\usepackage{txfonts}
\usepackage{stfloats}
\usepackage{bm}
\usepackage{cite}
\usepackage{cases}
\usepackage{subfig}

\usepackage{longtable}
\usepackage{multirow}

\usepackage{enumitem}
\usepackage{mathtools}
\usepackage{steinmetz}
\usepackage{tikz}
\usepackage{circuitikz}
\usepackage{verbatim}
\usepackage{tfrupee}
\usepackage[breaklinks=true]{hyperref}
\usepackage{graphicx}
\usepackage{tkz-euclide}
\usepackage{float}

\usetikzlibrary{calc,math}
\usepackage{listings}
    \usepackage{color}                                            %%
    \usepackage{array}                                            %%
    \usepackage{longtable}                                        %%
    \usepackage{calc}                                             %%
    \usepackage{multirow}                                         %%
    \usepackage{hhline}                                           %%
    \usepackage{ifthen}                                           %%
    \usepackage{lscape}     
\usepackage{multicol}
\usepackage{chngcntr}

\DeclareMathOperator*{\Res}{Res}

\renewcommand\thesection{\arabic{section}}
\renewcommand\thesubsection{\thesection.\arabic{subsection}}
\renewcommand\thesubsubsection{\thesubsection.\arabic{subsubsection}}

\renewcommand\thesectiondis{\arabic{section}}
\renewcommand\thesubsectiondis{\thesectiondis.\arabic{subsection}}
\renewcommand\thesubsubsectiondis{\thesubsectiondis.\arabic{subsubsection}}


\hyphenation{op-tical net-works semi-conduc-tor}
\def\inputGnumericTable{}                                 %%

\lstset{
%language=C,
frame=single, 
breaklines=true,
columns=fullflexible
}
\begin{document}


\newtheorem{theorem}{Theorem}[section]
\newtheorem{problem}{Problem}
\newtheorem{proposition}{Proposition}[section]
\newtheorem{lemma}{Lemma}[section]
\newtheorem{corollary}[theorem]{Corollary}
\newtheorem{example}{Example}[section]
\newtheorem{definition}[problem]{Definition}

\newcommand{\BEQA}{\begin{eqnarray}}
\newcommand{\EEQA}{\end{eqnarray}}
\newcommand{\define}{\stackrel{\triangle}{=}}
\newcommand\hlight[1]{\tikz[overlay, remember picture,baseline=-\the\dimexpr\fontdimen22\textfont2\relax]\node[rectangle,fill=blue!50,rounded corners,fill opacity = 0.2,draw,thick,text opacity =1] {$#1$};}
\bibliographystyle{IEEEtran}
\providecommand{\mbf}{\mathbf}
\providecommand{\pr}[1]{\ensuremath{\Pr\left(#1\right)}}
\providecommand{\qfunc}[1]{\ensuremath{Q\left(#1\right)}}
\providecommand{\sbrak}[1]{\ensuremath{{}\left[#1\right]}}
\providecommand{\lsbrak}[1]{\ensuremath{{}\left[#1\right.}}
\providecommand{\rsbrak}[1]{\ensuremath{{}\left.#1\right]}}
\providecommand{\brak}[1]{\ensuremath{\left(#1\right)}}
\providecommand{\lbrak}[1]{\ensuremath{\left(#1\right.}}
\providecommand{\rbrak}[1]{\ensuremath{\left.#1\right)}}
\providecommand{\cbrak}[1]{\ensuremath{\left\{#1\right\}}}
\providecommand{\lcbrak}[1]{\ensuremath{\left\{#1\right.}}
\providecommand{\rcbrak}[1]{\ensuremath{\left.#1\right\}}}
\theoremstyle{remark}
\newtheorem{rem}{Remark}
\newcommand{\sgn}{\mathop{\mathrm{sgn}}}
\providecommand{\abs}[1]{\left\vert#1\right\vert}
\providecommand{\res}[1]{\Res\displaylimits_{#1}} 
\providecommand{\norm}[1]{$\left\lVert#1\right\rVert$}
%\providecommand{\norm}[1]{\lVert#1\rVert}
\providecommand{\mtx}[1]{\mathbf{#1}}
\providecommand{\mean}[1]{E\left[ #1 \right]}
\providecommand{\fourier}{\overset{\mathcal{F}}{ \rightleftharpoons}}
%\providecommand{\hilbert}{\overset{\mathcal{H}}{ \rightleftharpoons}}
\providecommand{\system}{\overset{\mathcal{H}}{ \longleftrightarrow}}
	%\newcommand{\solution}[2]{\textbf{Solution:}{#1}}
\newcommand{\solution}{\noindent \textbf{Solution: }}
\newcommand{\cosec}{\,\text{cosec}\,}
\providecommand{\dec}[2]{\ensuremath{\overset{#1}{\underset{#2}{\gtrless}}}}
\newcommand{\myvec}[1]{\ensuremath{\begin{pmatrix}#1\end{pmatrix}}}
\newcommand{\mydet}[1]{\ensuremath{\begin{vmatrix}#1\end{vmatrix}}}
\numberwithin{equation}{subsection}
\makeatletter
\@addtoreset{figure}{problem}
\makeatother
\let\StandardTheFigure\thefigure
\let\vec\mathbf
\renewcommand{\thefigure}{\theproblem}
\def\putbox#1#2#3{\makebox[0in][l]{\makebox[#1][l]{}\raisebox{\baselineskip}[0in][0in]{\raisebox{#2}[0in][0in]{#3}}}}
     \def\rightbox#1{\makebox[0in][r]{#1}}
     \def\centbox#1{\makebox[0in]{#1}}
     \def\topbox#1{\raisebox{-\baselineskip}[0in][0in]{#1}}
     \def\midbox#1{\raisebox{-0.5\baselineskip}[0in][0in]{#1}}
\vspace{3cm}
\title{Assignment No.2}
\author{Valli Devi Bolla}
\maketitle
\newpage
\bigskip
\renewcommand{\thefigure}{\theenumi}
\renewcommand{\thetable}{\theenumi}
Download all python codes from 
\begin{lstlisting}
https://github.com/Vallidevibolla/Assignment-2/blob/main/main.tex
\end{lstlisting}
%
and latex-tikz codes from 
%
\begin{lstlisting}
https://github.com/Vallidevibolla/Assignment-2/blob/main/main.tex
\end{lstlisting}
%
Question taken from
\begin{lstlisting}
https://github.com/gadepall/ncert/blob/main/linalg/vectors/gvv_ncert_vectors.pdf- Q.no.2.18 
\end{lstlisting}
%
\section{Question No.2.18}

Consider the collision depicted in Fig. 2.18 to be between two billiard balls with equal masses m1=m2. The first ball is called the cue while the second ball is called the target. The billiard player wants to 'sink' the target ball in a corner pocket, which is at angle $\Phi$ = 37\degree. Assume that the collision is elastic and that friction and rotational motion are not important. Obtain $\theta$.

\numberwithin{figure}{section}
\begin{figure}[ht]
    \centering
    \includegraphics[width=\columnwidth]{download.png}
    \caption{Fig. 2.18}
    \label{Graphical solution}
\end{figure}

\section{Solution}
Given, two billiard balls with equal masses m1=m2

 $\therefore$ m1=m2=m
 
The first ball is called the \textbf{cue }while the second ball is called the \textbf{target}.

Figure shows that the cue is moving with initial velocity u1 towards target 

 $\therefore$ The initial velocity of cue = u1
 
 The initial velocity of target (Static) u2=0
 
 The cue moving with velocity collide the target thereby both balls get collide and travel in two directions with some velocity.
 
 $\therefore$ The final velocity of cue = v1
 
 The final velocity of target = v2
\section{Formula}
Momentum of the ball is given as P=mv
\begin{align}
    \boxed{m_1u_1+m_2u_2=m_1v_1+m_2v_2}  \\
since (u_2=0), also ( m_1=m_2=m )\\
    \boxed{u_1=v_1+v_2} 
\end{align}  
The energy of two balls after collision is given by Kinetic energy.
\textbf{K.E} =$\dfrac{1}{2}$mv^2 \\
 
$\dfrac{1}{2}$mu_1^2 = $\dfrac{1}{2}$mv_1^2 + $\dfrac{1}{2}$mv_2^2

\begin{align}
    \boxed{u_1^2 = v_1^2 + v_2^2} \\
   \implies(v_1+v_2)^2 = v_1^2+v_2^2\\
   \implies v_1^2+v_2^2+2v_1v_2 = v_1^2+v_2^2\\
   \implies2v_1v_2=0\\
   \implies v_1.v_2=0
\end{align}

since $\cos{90\degree}$=0\\
\implies v_1.v_2(\cos{\theta+\Phi})=\cos{90\degree}\\
Given, \Phi=37\degree
\implies \cos{(\theta+37\degree)}=\cos{90\degree}\\
\implies \theta+37\degree=90\degree\\
\boxed{\theta=53\degree}\\

$\therefore$ The angle of target was found to be $\theta$=53\degree

\end{document}
